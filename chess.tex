\documentclass{article}
\title{Explanation of madachess.py - autogenerated explanation}
\author{chatGPT}
\date{\today}

\begin{document}
\maketitle
\textbf{The script is a Pygame-based chess game that allows a user to play against a chess engine.} The script uses the Pygame library to draw the chess board and pieces on the screen. The chess engine used is Stockfish, which is a UCI-compatible chess engine. The script also uses the chess library to represent the game state and determine legal moves. Additionally, the script uses a socket to connect to a server, send the current game state and receive the move from the server.

The main functions of the script are:

\begin{itemize}
    \item \textbf{drawBoard(screen, board, squareSize, clicked)}: This function takes in the Pygame screen, a chess.Board object representing the current game state, the size of each square on the board, and the square that was last clicked by the user. It then uses a nested for loop to iterate through each square of the chess board, creating a rectangle for the current square and filling it with the appropriate color (light or dark). If there is a piece on the current square and the square has not been clicked by the user, it gets the corresponding image for the piece from the \textit{piecesImg} dictionary, scales it to the appropriate size, and draws it on the square.
    \item \textbf{drawHoveringPiece(screen, squareSize, pos, hoveringPiece)}: This function takes in the Pygame screen, the size of each square on the board, the position of the mouse cursor, and the chess piece that is being hovered over. It gets the corresponding image for the hovering piece from the \textit{piecesImg} dictionary, scales it to the appropriate size, and draws it on the square at the position of the mouse cursor.
    \item \textbf{main(screen, host\_ip, server\_port)}: This is the main function of the script. It initializes a chess.Board object, creates a socket connection and enters into an infinite loop. Inside the loop it gets the size of the Pygame screen and calculates the square size, creates a chess board rectangle, and checks if it is the turn of the chess engine. If it is the turn of the engine, it sends the last move made by the user to the server, receives the move made by the engine, and pushes that move on the board. Then it checks if the game is in checkmate, if so it prints "MATED!". The rest of the script continues to handle user input, drawing the board, and updating the game state.
\end{itemize}
\end{document}
